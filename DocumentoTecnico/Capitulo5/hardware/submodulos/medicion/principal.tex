\subsection{Submódulo Mediciones}
Correspondiente al submódulo de mediciones se implemento el uso de un sensor caudalimetro, el cual nos permite conocer la cantidad de flujo de liquido que pasa a través de él.
El valor de salida que nos devuelve el sensor al ingresar el liquido es una señal cuadrada, dicha señal tiene un voltaje pico a pico de 5 volts lo cual hace que el microcontrolador pueda trabajar con la señal sin necesidad de acondicionarla.
Es importante recalcar que el valor de la frecuencia de la señal es la que nos permitirá saber cual es la cantidad de flujo que esta pasando y que esta misma es el valor de entrada de el pin de interrupciones en el microcontrolador, dispositivo en el que se realizara el calculo de este valor.
