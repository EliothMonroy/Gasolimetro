\subsection{Submódulo Microcontrolador}
El microcontrolador ATMega16 es uno de los dispositivos fundamentales del presente proyecto, debido a que en él se centrara el calculo de la frecuencia de la señal entregada por el sensor.
Para realizar el calculo se utilizo el pin de interrupciones INT0, para obtener cada flanco de subida de la señal cuadrada, el cual se configura de la siguiente manera en el microcontrolador en codigo C.
\begin{lstlisting}[language= C]
void configurarPuertos(){
// Se configuran los puertos a utilizar
	cli(); // Deshabilita las interrupciones
	DDRA = (1<<0); // Habilitamos el PIN0 del puerto A como salida
	MCUCR |= (1<<ISC01) |(1<<ISC00); //INT0 flanco de subida
	GICR |= (1<<INT0); // INT0 habilitada
}
\end{lstlisting}
Posteriormente a habilitar el puerto de interrupciones del microcontrolador se realiza el calculo de la frecuencia de la señal, esto se hace por medio de la definición de frecuencia, la cual es: "La frecuencia es la cantidad de pulsos que se obtienen en un segundo".
\newline
El código que se encarga de hacer este calculo se muestra a continuación.
\begin{lstlisting}[language= C]
void obtenerFrecuencia(){
	int8_t frecuencia;
	numPulsos = 0;
	//Va la parte del contador de interrupciones
	sei();// Habilita las interrupciones
	_delay_ms(1000); // Esperamos un segundo
	cli();// detenemos las interrupciones
	frecuencia = numPulsos; // HZ (Cantidad de pulsos por segundo)
	//Enviamos los datos
	usar_transmit(frecuencia);
}

//Codigo que se ejecuta en cada interrupcion
ISR(INT0_vect){
	numPulsos++;	
	PORTA^=(1<<0);
}
\end{lstlisting}
Cada valor obtenido es enviado por bluetooth al dispositivo móvil, en el cual se realiza el calculo del flujo de combustible.

