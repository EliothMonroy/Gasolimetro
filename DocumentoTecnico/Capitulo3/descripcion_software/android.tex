\subsection{Android}
Es un sistema operativo móvil desarrollado por Google; está basado en Linux, que junto con aplicaciones middleware está enfocado para ser utilizado en dispositivos móviles como teléfonos inteligentes, tables, Google TV y otros dispositivos.
A continuación, en la Tabla \ref{Tabla_android} se muestra una comparación entre distintos sistemas operativos para dispositivos móviles.
\begin{table}[H]
	\centering
	\begin{tabular}{|M{3.2cm}|M{3cm}|M{3cm}|M{3cm}|}
		\hline 
		\textbf{Características} & \textbf{Android} & \textbf{IOS} & \textbf{Windows Phone} \\ \hline
		Núcleo & Linux & XNU & Windows NT \\ \hline
		Arquitectura soportada & ARM, MIPS, x86 & ARM & ARM, Microsoft XNA \\ \hline
		Programado & C, C++ y Java & C, C++, Objective-c y Swift & XNA, NET, C\#, C, C++ y VB.NET \\ \hline
		Entorno de desarrollo & Android Studio & Xcode & Visual Basic \\ \hline
		Conectividad & GSM/EDGE, IDEN, CDMA, EV-DO, UMTS, Bluetooth, WiFi, LITE, HSDPA, HSPA+, NFC y WiMAX & WiFi.802.11AC, Bluetooth LE & WiFi 802.11b/g y Bluetooth \\ \hline
		Sincronización con la nube & Google Drive & iCloud & SkyDrive \\ \hline
		Tienda de aplicaciones & Google Play & App Store & Windows Phone Store \\ \hline
		Almacenamiento de datos & SQLite & SQLite & SQLite \\ \hline
		Navegador web & Chrome & Safari & Internet Explorer \\ \hline
		Tipo de interfaz & Iconos y widgets & Iconos & Baldosas animadas \\ \hline
	\end{tabular}
	\caption{Tabla comparativa de dispositivos móviles}
	\label{Tabla_android}
\end{table}

Al comprar las características de las distintas tecnologías para dispositivos móviles, se decidió utilizar Android debido a que existe una gran parte de la población que cuenta con un dispositivo con este sistema operativo, asimismo nos brinda mayores posibilidades de conectividad y un entorno de desarrollo más accesible debido a que es una tecnología OpenSource.

