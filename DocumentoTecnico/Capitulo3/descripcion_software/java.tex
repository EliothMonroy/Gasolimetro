\subsection{Java}
Java es un lenguaje de programación y una plataforma informática comercializada por primera vez en 1995 por Sun Microsystems.\\
Es una tecnología que se usa para el desarrollo de aplicaciones que convierten a la Web en un elemento más interesante y útil.
Java permite jugar, cargar fotografías, chatear en línea, realizar visitas virtuales y utilizar servicios como, por ejemplo, cursos en línea, servicios bancarios en línea y mapas interactivos \citep{java}.
En la tabla \ref{tabla_lenguajes} se muestra una tabla comparativa entre tres distintos lenguajes de programación.
\begin{longtable}{|M{2.5cm}|M{3cm}|M{4.2cm}|M{4.3cm}|}
		\hline 
		\textbf{Lenguaje} & \textbf{Características} & \textbf{Ventajas} & \textbf{Desventajas} \\ \hline
		Java & Lenguaje orientado a objetos & \begin{itemize}
			\item Multiplataforma
			\item No existen problemas con la liberación de la memoria en el sistema.
			\item Cuenta con una amplia variedad de bibliotecas estándar.
		\end{itemize} &  \begin{itemize}
		\item Tiene un rendimiento menor comparado con los demas lenguajes de programación.
		\item Es necesario tener la maquina virtual de Java para poder ejecutar los programas.
		\item Es un lenguaje que evoluciona muy lentamente
	\end{itemize} \\ \hline
		C\# & Lenguaje de programación orientado a objetos & \begin{itemize}
			\item Declaración en el espacio de nombres
			\item Existe un rango más amplio y definido de tipos de datos.
			\item Propiedades: un objeto tiene intrínsecamente propiedades.
		\end{itemize} & \begin{itemize}
		\item Se necesita una versión reciente de VS.NET
		\item Se requiere Windows NT 4 o superior
	\end{itemize} \\ \hline
		Python & Lenguaje de programación interpretado y orientado a objetos & \begin{itemize}
			\item Tipado dinamico
			\item Facilidad de aprendizaje
			\item Sintaxis sencilla
		\end{itemize} & \begin{itemize}
		\item Al ser un lenguaje interpretado lo vuelve mas lento.
	\end{itemize} \\ \hline
	\caption{Tabla comparativa de diferentes lenguajes de programación}
	\label{tabla_lenguajes}
\end{longtable}
El lenguaje de programación que será utilizado en el presente proyecto será Java, debido a que el desarrollo de aplicaciónes moviles y web es mas sencillo. Ademas Java nos da la posibilidad de compilar una sola vez el proyecto y poder ejecutarlo en cualquier plataforma sin tener que recompilarlo de nuevo. Asimismo Java nos brinda seguridad, portabilidad y fiabilidad.