\chapter{Trabajo futuro}\label{chapter8}
Mejorar el diseño del sensor basado en la norma mexicana NOM-001-SCFI-1993, para asegurar la seguridad, de los usuarios, al momento de utilizar el sistema en el automóvil, así como reducir el tamaño del sensor y del circuito integrado al llevarlo a un ambiente productivo.
\\
Asimismo para tener un mejor resultado con respecto a la medición se utilizará un apego a la norma mexicana NOM-005-SCFI-2011 la cual se refiere al uso de instrumentos y sistemas para la medición de gasolina y otros combustibles líquidos,la cual brinda la aprobación del método de medición así como de una correcta verificación del flujo de combustible, y a su vez buscar tener una certificación de medición del sensor por parte del Centro Nacional de Metrología (CENAM).
\\
Como parte final, la información recabada por los usuarios del sistema puede ser utilizada para generar estadísticas especificas relacionada a la carga de combustible y de esta manera conocer como se comporta la Ciudad de México en el ámbito de carga de combustible.