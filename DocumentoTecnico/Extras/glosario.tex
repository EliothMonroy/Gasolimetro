\chapter{Glosario}\label{chapter9}
\noindent \textbf{API} Es un conjunto de reglas (código) y especificaciones que las aplicaciones pueden seguir para comunicarse entre ellas \citep{API}
\\
\textbf{Caracterización} Calcular por medio de medidas lo mas exactas posibles la ecuación característica del comportamiento del mismo, siendo esta la que determina la razón de cambio de la variable de salida respecto a la de entrada \cite{CARA}
\\
\textbf{Caudalímetro} Un caudalímetro es un instrumento de medida para la medición de caudal o gasto volumétrico de un fluido \citep{MarcoTeorico10}
\\
\textbf{Gateway} es un dispositivo, con frecuencia un ordenador, que permite interconectar redes con protocolos y arquitecturas diferentes a todos los niveles de comunicación. Su propósito es traducir la información del protocolo utilizado en una red al protocolo usado en la red de destino \cite{GATE}
\\
\textbf{Geolocalización} ULa tecnología del geoetiquetado se basa en la información posicional proporcionada por el sistema del sistema de posicionamiento global (GPS), y se transfiere como metadatos a los archivos que sean compatibles con este tipo de información \citep{GEO}
\\
\textbf{Interfaz} Dispositivo capaz de transformar las señales generadas por un aparato en señales comprensibles por otro.\citep{INTERFAZ}
\\
\textbf{Microservicios} Los microservicios son tanto un estilo de arquitectura como un modo de programar software. Con los microservicios, las aplicaciones se dividen en sus componentes más pequeños e independientes entre sí. \citep{MICROSERVICIOS}
\\
\textbf{Módulo} Es una porción de un programa de ordenador. \citep{MODULO}
\\
\textbf{Protocolo} Es un reglamento o una serie de instrucciones que se fijan por tradición o por convenio \citep{PROTOCOLO}
\\
\textbf{Requerimiento} Atributo necesario dentro de un sistema, que puede representar una capacidad, una característica o un factor de calidad del sistema \citep{REQUERIMIENTO}
\\
\textbf{Software} Soporte lógico de un sistema informático, que comprende el conjunto de los componentes lógicos necesarios que hacen posible la realización de tareas específicas\citep{SOFTWARE}


