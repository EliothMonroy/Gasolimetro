\section{Cambios en los cronogramas de actividades}
A continuación, se presentan los cambios realizados en el cronograma entregado durante el inicio del Trabajo Terminal (protocolo). Las subsiguientes secciones se dividen en cada uno de los integrantes del trabajo y las correcciones que hubo en sus cronogramas.
\subsection{Castillo Reyes Juan Daniel}
\begin{itemize}
	\item La actividad \textit{Pruebas unitarias y submódulo comunicación inalámbrica} fue eliminada debido a que está actividad fue designada a \textit{Monroy Martos Elioth} y \textit{Naranjo Miranda Javier Said}. Permitiendo así que, \textit{Castillo Reyes Juan Daniel} pudiera enfocarse a otras actividades relacionadas mayormente con el desarrollo del servidor web y la aplicación móvil.
	\item La actividad \textit{Evaluación de TT I} fue reacomodada para una mejor lectura del cronograma.
	\item La actividad \textit{Retroalimentación} fue añadida debido a los posibles cambios que pueden ser señalados por los Profesores Sinodales después de la Presentación de Trabajo Terminal I.
\end{itemize}
\subsection{Monroy Martos Elioth}
\begin{itemize}
	\item La actividad \textit{Retroalimentación} fue añadida debido a los posibles cambios que pueden ser señalados por los Profesores Sinodales después de la presentación de Trabajo Terminal I.
	\item Las actividades \textit{Análisis y diseño del submódulo acondicionamiento de la señal}, \textit{Implementación del submódulo acondicionamiento de la señal} y \textit{Pruebas unitarias submódulo acondicionamiento de la señal} fueron eliminadas debido a que el sensor y el microcontrolador seleccionados para la realización del Trabajo Terminal no requieren de un acondicionamiento de la señal (véanse las secciones \ref{sec:Sensor} y \ref{sec:micro}). Esto debido a que el voltaje que arroja el sensor es menor a 5 volts y el microcontrolador soporta como entrada el voltaje que manda el sensor.
\end{itemize}
\subsection{Naranjo Miranda Javier Said}
\begin{itemize}
	\item Las fechas de la actividades \textit{Análisis y diseño del submódulo medición}, \textit{Implementación del submódulo medición}, \textit{Pruebas unitarias submódulo medición}, \textit{Análisis y diseño del submódulo microcontrolador}, \textit{Implementación del submódulo microcontrolador}, \textit{Pruebas unitarias submódulo microcontrolador}, \textit{Análisis y diseño del submódulo comunicación inalámbrica}, \textit{Implementación del submódulo comunicación inalámbrica} y \textit{Pruebas unitarias del submódulo comunicación inalámbrica} fueron corregidas debido a que se encontraban en desorden, en general las fechas de las actividades fueron desplazadas una fila hacía arriba.
	\item La actividad \textit{Retroalimentación} fue añadida debido a los posibles cambios que pueden ser señalados por los Profesores Sinodales después de la presentación de Trabajo Terminal I.
	\item Las actividades \textit{Análisis y diseño del submódulo acondicionamiento de la señal}, \textit{Implementación del submódulo acondicionamiento de la señal} y \textit{Pruebas unitarias submódulo acondicionamiento de la señal} fueron eliminadas debido a que el sensor y el microcontrolador seleccionados para la realización del Trabajo Terminal no requieren de un acondicionamiento de la señal (véanse las secciones \ref{sec:Sensor} y \ref{sec:micro}). Esto debido a que el voltaje que arroja el sensor es menor a 5 volts y el microcontrolador soporta como entrada el voltaje que manda el sensor.
\end{itemize}