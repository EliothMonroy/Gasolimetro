\paragraph{SUB-U-CU5-Autenticar usuario}\label{SUB-U-CU5:Pruebas}
En la tabla \ref{pruebas_unitarias_sub-u-cu5} se pueden observar los resultados de las pruebas unitarias realizadas.
\begin{longtable}{|M{2cm}| M{4.5cm}|M{3cm}|M{3cm}|M{1.5cm}|}
	\hline
	\textbf{Prueba} & \textbf{Valores ingresados} & \textbf{Resultado esperado} & \textbf{Resultado obtenido} & \textbf{Prueba exitosa} \\ \hline
	Prueba de Trayectoria Principal 1 & 
	\begin{itemize}
		\item Correo electrónico: castilloreyesjuan@gmail.com
		\item Contraseña: alma123
	\end{itemize}
	& 
	Usuario autenticado exitosamente.
	&
	Usuario autenticado exitosamente.
	& Si \\ \hline

	Prueba de Trayectoria Principal 2 & 
	\begin{itemize}
		\item Correo electrónico: castilloreyesjuan@gmail.com
		\item Contraseña: alma1234
	\end{itemize}
	& 
	Usuario no puede autenticarse.
	&
	Mensaje datos incorrectos.
	& Si \\ \hline

	Prueba de Trayectoria Alternativa A 1 & 
	\begin{itemize}
		\item Correo electrónico: castilloreyesjuan@gmail.com
		\item Contraseña: ``''
	\end{itemize}
	& 
	Usuario no puede autenticarse.
	&
	Mensaje datos inválidos.
	& Si \\ \hline

	Prueba de Trayectoria Alternativa B 1 & 
	\begin{itemize}
		\item Correo electrónico: eliothmonroy@gmail.com
		\item Contraseña: ``contra123''
	\end{itemize}
	& 
	Usuario no puede autenticarse.
	&
	Mensaje cuenta no activa.
	& Si \\ \hline

	\caption{Resultados de las pruebas unitarias del caso de uso SUB-U-CU5-Autenticar usuario}
	\label{pruebas_unitarias_sub-u-cu5}
\end{longtable}