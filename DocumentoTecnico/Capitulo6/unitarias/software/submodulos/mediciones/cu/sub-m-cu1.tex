\paragraph{SUB-M-CU1-Calcular cantidad de combustible cargado}\label{SUB-M-CU1:Pruebas}
En la tabla \ref{pruebas_unitarias_sub-m-cu1} se pueden observar los resultados de las pruebas unitarias realizadas.
\begin{longtable}{|M{2cm}| M{4.5cm}|M{3cm}|M{3cm}|M{1.5cm}|}
	\hline
	\textbf{Prueba} & \textbf{Valores ingresados} & \textbf{Resultado esperado} & \textbf{Resultado obtenido} & \textbf{Prueba exitosa} \\ \hline
	Prueba de Trayectoria Principal 1& Valor de las tramas: 17-20-10-20-20-13-20-15-16-17-20-20-20-20-0-0-0-0-0-0-0-0-0-0-0-0
	& Total de litros: 5 & Total de litros: 5.1 & Si \\ \hline
	Prueba de Trayectoria Principal 1& Valor de las tramas: 10-10-10-20-15-13-20-15-8-17-20-23-20-21-0-0-0-0-0-0-0-0-0-0-0-0
	& Total de litros: 6 & Total de litros: 6.14 & Si \\ \hline
	\caption{Resultados de las pruebas unitarias del caso de uso SUB-M-CU1-Calcular cantidad de combustible cargado}
	\label{pruebas_unitarias_sub-m-cu1}
\end{longtable}

