\paragraph{SUB-M-CU1.1.1-Almacenar medición}\label{SUB-M-CU1.1.1:Pruebas}
En la tabla \ref{pruebas_unitarias_sub-m-cu1.1.1} se pueden observar los resultados de las pruebas unitarias realizadas.
\begin{longtable}{|M{2cm}| M{4.5cm}|M{3cm}|M{3cm}|M{1.5cm}|}
	\hline
	\textbf{Prueba} & \textbf{Valores ingresados} & \textbf{Resultado esperado} & \textbf{Resultado obtenido} & \textbf{Prueba exitosa} \\ \hline
	Prueba de Trayectoria Principal 1& 
	\begin{itemize}
		\item Cantidad de combustible cargada al automóvil: 5 litros.
		\item Fecha y hora de carga: 01/05/19 12:00
		\item Cantidad de gasolina que el usuario ingreso: 5 litros.
		\item idSensor: 1.
	\end{itemize}
	& Mensaje: ``Medición registrada exitosamente'' & Se recibió el mensaje: ``Medición registrada exitosamente'' & Si \\ \hline
	Prueba de Trayectoria Principal 2& 
	\begin{itemize}
		\item Cantidad de combustible cargada al automóvil: 10 litros.
		\item Fecha y hora de carga: 01/05/19 12:15
		\item Cantidad de gasolina que el usuario ingreso: 10 litros.
		\item idSensor: 2.
	\end{itemize}
	& Mensaje: ``Medición registrada exitosamente'' & Se recibió el mensaje: ``Medición registrada exitosamente'' & Si \\ \hline
	\caption{Resultados de las pruebas unitarias del caso de uso SUB-M-CU1.1.1-Almacenar medición}
	\label{pruebas_unitarias_sub-m-cu1.1.1}
\end{longtable}