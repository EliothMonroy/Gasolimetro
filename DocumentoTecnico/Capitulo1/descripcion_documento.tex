%%%%%%%%%%%%%%%%%%%%%%%%%%%%%%%%%%%%%%%%%%%%%%%%%%%%%%%%%%%%%%%%%%%%%%%%%
%                    Descripción del documento                           %
%%%%%%%%%%%%%%%%%%%%%%%%%%%%%%%%%%%%%%%%%%%%%%%%%%%%%%%%%%%%%%%%%%%%%%%%%
\chapter*{Descripción del documento}
\addcontentsline{toc}{chapter}{Descripción del documento}  
%\section{Descripción del documento}
En el \textit{Capítulo \ref{chapter1} Introducción}, se expone de manera breve la motivación por la cual surgió el presente trabajo terminal, exponiendo el planteamiento del problema y los objetivos tanto generales como específicos que han sido planteados para el mismo.\\
En los capítulos siguientes se expresará a mayor profundidad el trabajo terminal, en el \textit{Capítulo \ref{chapter2} Marco conceptual} se presenta una referencia general, al ámbito bajo el cual es desarrollado el sistema, dando así, una introducción a los temas que son de interés para la correcta compresión de las diversas temáticas que abarca el trabajo terminal.\\
En el \textit{Capítulo \ref{chapter3} Análisis del sistema} se redactan los distintos factores tomados en cuenta para las selecciones de tecnología, se explican los motivos por los cuales fueron usados ciertos dispositivos de hardware y ciertas tecnologías, además de que se presenta una sustentación del sistema bajo los análisis de factibilidad y el análisis de riesgos.\\
El \textit{Capítulo \ref{chapter4} Diseño del sistema} se presenta el diseño de los distintos submódulos que componen al sistema, mostrando para la parte de hardware, los distintos diagramas de los circuitos que son usados, además, para la parte de software, se muestran los diagramas UML que permiten la construcción del sistema, entre los cuales se incluyen, diagramas de casos de uso, diagramas de secuencia y diagrama de clases. Finalmente, en este capítulo se muestra la arquitectura del sistema y el diagrama relacional de la base de datos.\\
Por último, el desarrollo del trabajo se explica en el \textit{Capítulo \ref{chapter5} Desarrollo del sistema} así como las pruebas para el mismo se exponen en el \textit{Capítulo \ref{chapter6} Pruebas del sistema}.\\
Las conclusiones obtenidas de este trabajo terminal se muestran en el \textit{Capítulo \ref{chapter7} Conclusiones} y el trabajo futuro en el \textit{Capítulo \ref{chapter8} Trabajo futuro}.\\