%%%%%%%%%%%%%%%%%%%%%%%%%%%%%%%%%%%%%%%%%%%%%%%%%%%%%%%%%%%%%%%%%%%%%%%%%
%                           Presentación                                %
%%%%%%%%%%%%%%%%%%%%%%%%%%%%%%%%%%%%%%%%%%%%%%%%%%%%%%%%%%%%%%%%%%%%%%%%%

\section{Presentación} % section headings are printed smaller than chapter names
El suministro de gasolina es una actividad que se realiza diariamente en miles de gasolineras de todo México, esta actividad consiste en la transmisión de gasolina al contenedor de combustible de un automóvil a través de una máquina dispensadora, dicho dispositivo registra la cantidad de gasolina suministrada (medida en litros) y muestra el equivalente en pesos a pagar según el precio del litro gasolina.
Dicha actividad es de suma importancia, ya que gran parte de las actividades económicas y personales en nuestro país dependen del uso de un vehículo cuyo funcionamiento requiera algún tipo de combustible. El proceso de suministro de gasolina depende de diversos factores, como lo son una serie de dispositivos electrónicos y mecánicos conectados entre sí, además de la interacción con la persona despachadora de la gasolina, todos estos factores influyen en la precisión con la que los litros de gasolina son suministrados a un vehículo.
\\
En los últimos años, la Profeco (Procuraduría Federal del Consumidor) ha detectado nuevas modalidades para el robo de gasolina, la mayoría de estas, sucede durante el proceso de recarga o suministro de gasolina, lo cuál, muchas veces suele pasar como un fenómeno desapercibido para el usuario final. 
Este problema, cobra especial relevancia en un país de consumo elevado de gasolina como México. Petróleos Mexicanos (PEMEX) informó que durante el primer semestre del año 2016 el consumo de gasolina fue de 812 mil barriles por día. Lo anterior equivale a 129 millones de litros diarios, de los cuales, 78 por ciento corresponde al tipo Magna y 22 por ciento al tipo Premium, según explica dicho órgano en su cuenta oficial de Twitter \cite{Pre1}.
Las cifras anteriores ubican a México entre los primeros consumidores de gasolina a nivel mundial. Además, en el año 2014 México ocupo la cuarta posición a nivel mundial en consumo de gasolina al día, solo por debajo de Estados Unidos, Japón y Canadá \citep{Pre2}.
\\
Otro factor importante en el consumo de gasolina es el precio, según el sitio Global Petrol Prices en el año 2018 México se encuentra entre los países latinoamericanos con los precios más elevados para el litro de gasolina ubicándose en 1.01 dólares por litro al término del primer semestre \citep{Pre3}.
\\
La estadística anterior cobra mayor relevancia cuándo se estudian los ingresos económicos del mexicano. Los mexicanos gastan en promedio un 3.38 por ciento de sus ingresos, unos 5 mil 336 pesos, en comprar 358.94 litros de gasolina al año, que es el promedio que utiliza un conductor en el país \citep{Pre4}.
\\
Basándonos en las estadísticas mencionadas con anterioridad observamos que el consumo de gasolina en México es una actividad de suma importancia y que afecta en buena parte las economías de las familias mexicanas. Dada la importancia económica de esta actividad las gasolineras cuentan con mecanismos de revisión que corroboran que el suministro de gasolina se realice de manera correcta en las gasolineras de país. Lamentablemente, muchas irregularidades se han presentado en los últimos años en diversas gasolineras, la mayoría de ellas tiene que ver con la exactitud al momento de despechar litros de combustible.
\\
Durante el 2014, la Procuraduría Federal del Consumidor –Profeco- revisó 1,792 gasolineras, de las cuales, el 56\% (1,017 estaciones) tuvieron alguna irregularidad. Esta cifra es sin contar a las 233 gasolineras que se negaron a ser verificadas y mejor pagaron la multa correspondiente.
En lo que va del año 2018, la Profeco ha revisado 400 estaciones de servicio, de las cuales 68\% (274 gasolineras) fueron inmovilizadas no sólo por no despachar completo sino por no contar con las señalizaciones adecuadas. Asimismo, 39 gasolineras se negaron a ser revisadas por lo que pagaron la multa de \$250,000 pesos \citep{Pre5}.
\\
Actualmente, la Profeco posee una lista negra con las gasolineras en las que se han encontrado irregularidades, sin embargo, dicha lista es actualizada después del periodo de revisiones lo cuál decrementa la exactitud de la lista debido a un periodo largo de actualización, además, cómo ya se mencionó existen gasolineras que evitan una revisión mediante el pago de una multa, lo cuál impide que exista una clasificación correcta de las gasolineras.
\\
La propuesta de solución planteada en este trabajo terminal consiste en la creación de una aplicación móvil que permita realizar una mejor clasificación de las gasolineras de acuerdo con su exactitud al momento de cargar gasolina. Dicha clasificación será actualizada constantemente con la ayuda de un sensor que medirá la cantidad de litros ingresados al vehículo, de esta forma, pondrán ser censadas todas aquellas gasolineras que evitan una revisión de la Profeco mediante el pago de una multa. Finalmente, la clasificación de las gasolineras no dependerá de una sola revisión, sino que será retroalimentada por las mediciones constantes de cada sensor presente en un automóvil al momento de cargar gasolina, lo cuál incrementa la exactitud en el algoritmo de clasificación.