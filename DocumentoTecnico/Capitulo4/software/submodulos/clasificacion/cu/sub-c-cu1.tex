\subsubsection{SUB-C-CU1-Generar clasificación}\label{SUB-C-CU1}
Proceso que se ejecuta cada determinado tiempo según lo definido en la regla de negocio \ref{RN7}, para actualizar la clasificación de las gasolineras.

\begin{longtable}{|J{5cm}|J{10.3cm}|}
	\hline
	\textbf{Nombre del caso de uso} &
		SUB-C-CU1-Generar clasificación \\ \hline
	\textbf{Objetivo} &
		Actualizar la clasificación de gasolineras con los nuevos valores que se han medido. \\ \hline
	\textbf{Actores} &
		Proceso en el tiempo \\ \hline 
	\textbf{Disparador} & 
		Se cumple el tiempo establecido. \\ \hline 
	\textbf{Entradas} & 
		\begin{itemize}
				\item Listado de gasolineras.
				\item Mediciones.
		\end{itemize}\\ \hline 
	\textbf{Salidas} & 
		\begin{itemize}
			\item Listado de gasolineras actualizado.
		\end{itemize} \\ \hline
	\textbf{Precondiciones} &
		Ninguna.\\ \hline
	\textbf{Postcondiciones} &
		\begin{itemize}
			\item La clasificación actualizada puede ser consultada por los usuarios.
		\end{itemize} \\ \hline
	\textbf{Condiciones de término} & 
		\begin{itemize}
			\item Se actualiza la clasificación de gasolineras.
		\end{itemize} \\ \hline 
	\textbf{Prioridad} & 
		Alta. \\ \hline
	\textbf{Errores} & Ninguno.
		% \begin{itemize}
		% 	\item \label{SUB-M-CU1:Error1} Error 1: .
		% \end{itemize} 
		\\ \hline
	\textbf{Reglas de negocio} & \ref{RN7}.
		% \begin{itemize}
		% 	\item 
		% \end{itemize}
		 \\ \hline
	% \caption{}
	%\label{desc:SUB-M-CU1}
\end{longtable}

\paragraph{Trayectoria principal}
	\begin{enumerate}
		\item {[Sistema]} Verifica que se haya cumplido el tiempo especifico para que se actualice la clasificación de gasolineras con base en lo establecido en la regla de negocio \ref{RN7}.
		\item {[Sistema]} Obtiene el listado de gasolineras registradas.
		\item {[Sistema]} Obtiene las mediciones registradas para cada una de las gasolineras obtenidas.
		\item {[Sistema]} Obtiene la media de mediciones para cada una de las gasolineras obtenidas.
		\item {[Sistema]} Persiste la clasificación calculada para cada una de las gasolineras.
	\end{enumerate}
	Fin del caso de uso.

% \paragraph{Trayectoria alternativa A} \label{SUB-M-CU1.1:TA}
% 	El actor no se encuentra usando la aplicación móvil.
% 	\begin{enumerate}[label=A\arabic*.]
% 		\item {[Sistema]} Muestra una notificación al actor como la que se observa en la pantalla \hyperref[fig:sub-m-iu1.1.a]{SUB-M-IU1.1-Confirmar medición (a)}.
% 		\item {[Actor]} Presiona la notificación.
% 		\item {[Sistema]} Continúa en el paso \ref{SUB-M-CU1.1:Pantalla} de la Trayectoria Principal.
% 	\end{enumerate}
% 	Fin de la trayectoria alternativa.

% \paragraph{Puntos de extensión} \label{SUB-M-CU1.1:P}
% \begin{enumerate}[label=PE\arabic*.]
% 	\item Caso de uso \hyperref[SUB-M-CU1.1.3]{SUB-M-CU1.1.3-Obtener insignia} en el paso \ref{SUB-M-CU1.1:Boton} de la Trayectoria principal.
% 	\item Caso de uso \hyperref[SUB-M-CU1.1.4]{SUB-M-CU1.1.4-Asignar insignia a gasolinera} en el paso \ref{SUB-M-CU1.1:Boton} de la Trayectoria principal.
% 	\item Caso de uso \hyperref[SUB-M-CU1.1.5]{SUB-M-CU1.1.5-Especificar bomba} en el paso \ref{SUB-M-CU1.1:Boton} de la Trayectoria principal.
% \end{enumerate}
