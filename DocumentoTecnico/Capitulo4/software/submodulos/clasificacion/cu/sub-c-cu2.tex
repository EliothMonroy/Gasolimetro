\subsubsection{SUB-C-CU2-Consultar mapa}\label{SUB-C-CU2}
Cuando el actor abre la aplicación móvil, se le muestra en un mapa su ubicación, junto con los iconos correspondientes a las gasolineras, que se encuentren en el radio definido por la regla de negocio \ref{RN9}.

\begin{longtable}{|J{5cm}|J{10.3cm}|}
	\hline
	\textbf{Nombre del caso de uso} &
		SUB-C-CU2-Consultar mapa \\ \hline
	\textbf{Objetivo} &
		Mostrar al actor las gasolineras cercanas a él, además de la clasificación de las mismas. \\ \hline
	\textbf{Actores} &
		\begin{itemize}
			\item Cliente.
			\item Usuario no registrado.
			\item Administrador.
		\end{itemize}
		 \\ \hline 
	\textbf{Disparador} & 
		El actor ingresa a la aplicación móvil. \\ \hline
	\textbf{Entradas} & 
		\begin{itemize}
				\item Ubicación del actor.
		\end{itemize}\\ \hline 
	\textbf{Salidas} & 
		\begin{itemize}
			\item Mapa que muestra las gasolineras cercanas a él, al igual que la clasificación y dirección de las mismas.
		\end{itemize} \\ \hline
	\textbf{Precondiciones} &
		Ninguna.\\ \hline
	\textbf{Postcondiciones} & El actor puede seleccionar una gasolinera de las que le aparecen en el mapa.
		% \begin{itemize}
		% 	\item 
		% \end{itemize} 
		\\ \hline
	\textbf{Condiciones de término} & Se muestra el mapa con la clasificación de gasolinera al actor.
		% \begin{itemize}
		% 	\item 
		% \end{itemize}
		\\ \hline 
	\textbf{Prioridad} & 
		Alta. \\ \hline
	\textbf{Errores} & Ninguno.
		% \begin{itemize}
		% 	\item \label{SUB-M-CU1:Error1} Error 1: .
		% \end{itemize} 
		\\ \hline
	\textbf{Reglas de negocio} & 
		\begin{itemize}
			\item \ref{RN8}.
			\item \ref{RN9}.
		\end{itemize}
		 \\ \hline
	% \caption{}
	%\label{desc:SUB-M-CU1}
\end{longtable}

\paragraph{Trayectoria principal}
	\begin{enumerate}
		\item {[Actor]} Abre la aplicación móvil.
		\item {[Sistema]} Obtiene la ubicación del usuario.
		\item {[Sistema]} Obtiene las gasolineras que se encuentran en un radio establecido por la regla de negocio \ref{RN8} del API de Google Maps.
		\item {[Sistema]} Compara las gasolineras obtenidas en el paso anterior con las gasolineras registradas en el sistema.
		\item {[Sistema]} De las gasolineras cercanas al actor, obtiene la clasificación, nombre y dirección de las mismas.
		\item {[Sistema]} Muestra el icono en el mapa para cada una de las gasolineras según lo indicado por la regla de negocio \ref{RN9}.
		\item \label{SUB-C-CU2:Pantalla} {[Sistema]} Muestra la información obtenida en la pantalla \hyperref[fig:sub-c-iu2]{SUB-C-IU2-Consultar mapa}.
	\end{enumerate}
	Fin del caso de uso.

\paragraph{Puntos de extensión} \label{SUB-C-CU2:PE}
\begin{enumerate}[label=PE\arabic*.]
	\item Caso de uso \hyperref[SUB-C-CU2.1]{SUB-C-CU2.1-Realizar recorrido} en el paso \ref{SUB-C-CU2:Pantalla} de la Trayectoria principal.
	\item Caso de uso \hyperref[SUB-C-CU2.2]{SUB-C-CU2.2-Consultar gasolinera} en el paso \ref{SUB-C-CU2:Pantalla} de la Trayectoria principal.
\end{enumerate}
