\subsubsection{SUB-U-CU10-Registrar automóvil}\label{SUB-U-CU10}
Un Cliente puede registrar un automóvil desde el que estará haciendo mediciones con un sensor.

\begin{longtable}{|J{5cm}|J{10.3cm}|}
	\hline
	\textbf{Nombre del caso de uso} &
		SUB-U-CU10-Registrar automóvil \\ \hline
	\textbf{Objetivo} &
		Permitir a un Cliente registrar un automóvil. \\ \hline
	\textbf{Actores} &
		Cliente \\ \hline 
	\textbf{Disparador} & 
		El actor requiere registrar un automóvil. \\ \hline 
	\textbf{Entradas} & 
		\begin{itemize}
				\item Nombre del automóvil.
				\item Modelo del automóvil.
				\item Marca del automóvil.
		\end{itemize}\\ \hline 
	\textbf{Salidas} & Mensaje de éxito.
		% \begin{itemize}
		% 	\item 
		% \end{itemize} 
		\\ \hline
	\textbf{Precondiciones} & Ninguna.
		\\ \hline
	\textbf{Postcondiciones} & El automóvil registrado puede ser asociado a un sensor.
		% \begin{itemize}
		% 	\item El actor puede verificar su cuenta.
		% \end{itemize} 
		\\ \hline
	\textbf{Condiciones de término} & Se muestra el mensaje de éxito al actor.
		% \begin{itemize}
		% 	\item Se muestra el mensaje de éxito al actor.
		% \end{itemize} 
		\\ \hline 
	\textbf{Prioridad} & 
		Baja. \\ \hline
	\textbf{Errores} & Ninguno.
		% \begin{itemize}
		% 	\item \label{SUB-M-CU1:Error1} Error 1: .
		% \end{itemize} 
		\\ \hline
	\textbf{Reglas de negocio} & \ref{RN11}.
		% \begin{itemize}
		% 	\item \ref{RN1}.
		% \end{itemize}
		 \\ \hline
	% \caption{}
	%\label{desc:SUB-M-CU1}
\end{longtable}

\paragraph{Trayectoria principal}
	\begin{enumerate}
		\item {[Actor]} Selecciona la opción \textit{Automóviles} del menú \hyperref[fig:menu-cliente]{Menú para Cliente}.
		\item {[Sistema]} Obtiene los automóviles registrados por el actor.\hyperref[SUB-U-CU10:TA]{Trayectoria alternativa A}
		\item {[Sistema]} Muestra la lista de automóviles asociados al actor.
		\item \label{SUB-U-CU10:Agregar} {[Actor]} Presiona el icono \textit{Agregar automóvil}.
		\item {[Sistema]} Muestra la pantalla \hyperref[fig:sub-u-iu10]{SUB-U-IU10-Registrar automóvil}.
		\item \label{SUB-U-CU10:Ingresar} {[Actor]} Ingresa la información solicitada por la pantalla.
		\item {[Actor]} Presiona el botón \textit{Registrar}.
		\item {[Sistema]} Verifica que la información ingresada por el usuario sea válida la información ingresada por el usuario, según lo indicado por la regla de negocio \ref{RN11}.\hyperref[SUB-U-CU10:TB]{Trayectoria alternativa B}
		\item {[Sistema]} Persiste la información ingresada.
		\item {[Sistema]} Muestra un mensaje de éxito indicando al actor que la acción fue realizada exitosamente.
		\item \label{SUB-U-CU10:Pantalla} {[Sistema]} Muestra la pantalla \hyperref[fig:sub-c-iu2]{SUB-C-IU2-Consultar mapa}.
	\end{enumerate}
	Fin del caso de uso.

\paragraph{Trayectoria alternativa A} \label{SUB-U-CU10:TA}
	El actor no tiene ningún automóvil registrado.
	\begin{enumerate}[label=A\arabic*.]
		\item {[Sistema]} Muestra un mensaje al actor indicando que aún no tiene automóviles registrados.
		\item {[Sistema]} Continúa en el paso \ref{SUB-U-CU10:Agregar} de la Trayectoria Principal.
	\end{enumerate}
	Fin de la trayectoria alternativa.

\paragraph{Trayectoria alternativa B} \label{SUB-U-CU10:TB}
	La información ingresada por el actor no es válida.
	\begin{enumerate}[label=A\arabic*.]
		\item {[Sistema]} Muestra al actor un mensaje indicando que la información ingresada no es válida.
		\item {[Sistema]} Continúa en el paso \ref{SUB-U-CU10:Ingresar} de la Trayectoria Principal.
	\end{enumerate}
	Fin de la trayectoria alternativa.

% \paragraph{Puntos de extensión} \label{SUB-M-CU1.1:P}
% \begin{enumerate}[label=PE\arabic*.]
% 	\item Caso de uso \hyperref[SUB-M-CU1.1.3]{SUB-M-CU1.1.3-Obtener insignia} en el paso \ref{SUB-M-CU1.1:Boton} de la Trayectoria principal.
% 	\item Caso de uso \hyperref[SUB-M-CU1.1.4]{SUB-M-CU1.1.4-Asignar insignia a gasolinera} en el paso \ref{SUB-M-CU1.1:Boton} de la Trayectoria principal.
% 	\item Caso de uso \hyperref[SUB-M-CU1.1.5]{SUB-M-CU1.1.5-Especificar bomba} en el paso \ref{SUB-M-CU1.1:Boton} de la Trayectoria principal.
% \end{enumerate}
