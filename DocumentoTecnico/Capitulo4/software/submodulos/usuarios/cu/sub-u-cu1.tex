\subsubsection{SUB-U-CU1-Registrar usuario}\label{SUB-U-CU1}
Un usuario no registrado puede registrarse al sistema ingresando información como su nombre y correo electrónico, esto con la finalidad de poder registrar un sensor y acceder a la funcionalidad completa de la aplicación móvil.

\begin{longtable}{|J{5cm}|J{10.3cm}|}
	\hline
	\textbf{Nombre del caso de uso} &
		SUB-U-CU1-Registrar usuario \\ \hline
	\textbf{Objetivo} &
		Permitir a un usuario no registrado crear una cuenta en el sistema. \\ \hline
	\textbf{Actores} &
		Usuario no registrado \\ \hline 
	\textbf{Disparador} & 
		El actor requiere registrarse en el sistema. \\ \hline 
	\textbf{Entradas} & 
		\begin{itemize}
				\item Nombre.
				\item Apellido paterno.
				\item Apellido materno.
				\item Correo electrónico.
				\item Contraseña.
		\end{itemize}\\ \hline 
	\textbf{Salidas} & 
		\begin{itemize}
			\item Mensaje de éxito.
			\item Mensaje enviado por correo electrónico para que el actor verifique su cuenta.
		\end{itemize} \\ \hline
	\textbf{Precondiciones} &
		El correo electrónico no debe estar registrado en el sistema.\\ \hline
	\textbf{Postcondiciones} & El actor puede verificar su cuenta.
		% \begin{itemize}
		% 	\item 
		% \end{itemize}
		\\ \hline
	\textbf{Condiciones de término} & Se muestra el mensaje de éxito al actor.
		% \begin{itemize}
		% 	\item 
		% \end{itemize}
		\\ \hline 
	\textbf{Prioridad} & 
		Media. \\ \hline
	\textbf{Errores} & Ninguno.
		% \begin{itemize}
		% 	\item \label{SUB-M-CU1:Error1} Error 1: .
		% \end{itemize} 
		\\ \hline
	\textbf{Reglas de negocio} & \ref{RN10}.
		% \begin{itemize}
		% 	\item \ref{RN1}.
		% \end{itemize}
		 \\ \hline
	% \caption{}
	%\label{desc:SUB-M-CU1}
\end{longtable}

\paragraph{Trayectoria principal}
	\begin{enumerate}
		\item {[Actor]} Selecciona la opción \textit{Regístrate} del menú \hyperref[fig:menu-usuario]{Menú para Usuario no registrado}.
		\item {[Sistema]} Muestra la pantalla \hyperref[fig:sub-u-iu1]{SUB-U-IU1-Registrar usuario}.
		\item \label{SUB-U-CU1:Ingresar} {[Actor]} Ingresa la información solicitada por la pantalla.
		\item {[Actor]} Presiona el botón \textit{Registrarse}.
		\item {[Sistema]} Verifica que la información ingresada por el actor sea válida, según lo indicado por la regla de negocio \ref{RN11}.\hyperref[SUB-U-CU1:TA]{Trayectoria alternativa A}
		\item {[Sistema]} Envía un correo electrónico solicitando la confirmación de su cuenta al correo ingresado según lo establecido en la regla de negocio \ref{RN10}.
		\item {[Sistema]} Persiste la información ingresada por el actor.
		\item {[Sistema]} Muestra un mensaje indicando al actor que debe verificar su cuenta.
		\item \label{SUB-U-CU1:Pantalla} {[Sistema]} Muestra la pantalla \hyperref[fig:sub-c-iu2]{SUB-C-IU2-Consultar mapa}.
	\end{enumerate}
	Fin del caso de uso.

\paragraph{Trayectoria alternativa A} \label{SUB-U-CU1:TA}
	La información ingresada por el actor no es válida.
	\begin{enumerate}[label=A\arabic*.]
		\item {[Sistema]} Muestra al actor un mensaje indicando que la información ingresada no es válida.
		\item {[Sistema]} Continúa en el paso \ref{SUB-U-CU1:Ingresar} de la Trayectoria Principal.
	\end{enumerate}
	Fin de la trayectoria alternativa.

% \paragraph{Trayectoria alternativa A} \label{SUB-M-CU1.1:TA}
% 	El actor no se encuentra usando la aplicación móvil.
% 	\begin{enumerate}[label=A\arabic*.]
% 		\item {[Sistema]} Muestra una notificación al actor como la que se observa en la pantalla \hyperref[fig:sub-m-iu1.1.a]{SUB-M-IU1.1-Confirmar medición (a)}.
% 		\item {[Actor]} Presiona la notificación.
% 		\item {[Sistema]} Continúa en el paso \ref{SUB-M-CU1.1:Pantalla} de la Trayectoria Principal.
% 	\end{enumerate}
% 	Fin de la trayectoria alternativa.

\paragraph{Puntos de extensión} \label{SUB-U-CU1:P}
\begin{enumerate}[label=PE\arabic*.]
	\item Caso de uso \hyperref[SUB-U-CU1.1]{SUB-U-CU1.1-Verificar cuenta} en el paso \ref{SUB-U-CU1:Pantalla} de la Trayectoria principal.
\end{enumerate}
