\subsubsection{SUB-U-CU11-Editar automóvil}\label{SUB-U-CU11}
El Cliente que tenga automóviles registrados puede editar la información de los mismo, así como asociarlo a un sensor en especifico en caso de tener al menos uno registrado.

\begin{longtable}{|J{5cm}|J{10.3cm}|}
	\hline
	\textbf{Nombre del caso de uso} &
		SUB-U-CU11-Editar automóvil \\ \hline
	\textbf{Objetivo} &
		Permitir a un Cliente editar un automóvil. \\ \hline
	\textbf{Actores} &
		Cliente \\ \hline 
	\textbf{Disparador} & 
		El actor requiere editar un automóvil. \\ \hline 
	\textbf{Entradas} & 
		\begin{itemize}
				\item Nombre del automóvil.
				\item Modelo del automóvil.
				\item Marca del automóvil.
		\end{itemize}\\ \hline 
	\textbf{Salidas} & Mensaje de éxito.
		% \begin{itemize}
		% 	\item 
		% \end{itemize} 
		\\ \hline
	\textbf{Precondiciones} & El actor tiene registrado al menos un automóvil.
		\\ \hline
	\textbf{Postcondiciones} & La información del automóvil es actualizada.
		% \begin{itemize}
		% 	\item El actor puede verificar su cuenta.
		% \end{itemize} 
		\\ \hline
	\textbf{Condiciones de término} & Se muestra el mensaje de éxito al actor.
		% \begin{itemize}
		% 	\item Se muestra el mensaje de éxito al actor.
		% \end{itemize} 
		\\ \hline 
	\textbf{Prioridad} & 
		Baja. \\ \hline
	\textbf{Errores} & Ninguno.
		% \begin{itemize}
		% 	\item \label{SUB-M-CU1:Error1} Error 1: .
		% \end{itemize} 
		\\ \hline
	\textbf{Reglas de negocio} &  \ref{RN11}.
		% \begin{itemize}
		% 	\item \ref{RN1}.
		% \end{itemize}
		 \\ \hline
	% \caption{}
	%\label{desc:SUB-M-CU1}
\end{longtable}

\paragraph{Trayectoria principal}
	\begin{enumerate}
		\item {[Actor]} Presiona el botón \textit{Editar} de un automóvil.
		\item {[Sistema]} Obtiene la información del automóvil seleccionado.
		\item {[Sistema]} Muestra la pantalla \hyperref[fig:sub-u-iu11]{SUB-U-IU11-Editar automóvil}.
		\item \label{SUB-U-CU11:Ingresar} {[Actor]} Ingresa la información solicitada por la pantalla.
		\item {[Actor]} Presiona el botón \textit{Confirmar}.
		\item {[Sistema]} Verifica que la información ingresada por el actor sea válida, según lo indicado por la regla de negocio \ref{RN11}.\hyperref[SUB-U-CU11:TA]{Trayectoria alternativa A}
		\item {[Sistema]} Persiste la información ingresada.
		\item {[Sistema]} Muestra un mensaje de éxito indicando al actor que la acción fue realizada exitosamente.
		\item \label{SUB-U-CU11:Pantalla} {[Sistema]} Muestra la pantalla \hyperref[fig:sub-u-iu12]{SUB-U-IU12-Consultar automóvil}.
	\end{enumerate}
	Fin del caso de uso.

\paragraph{Trayectoria alternativa A} \label{SUB-U-CU11:TA}
	La información ingresada por el actor no es válida.
	\begin{enumerate}[label=A\arabic*.]
		\item {[Sistema]} Muestra al actor un mensaje indicando que la información ingresada no es válida.
		\item {[Sistema]} Continúa en el paso \ref{SUB-U-CU11:Ingresar} de la Trayectoria Principal.
	\end{enumerate}
	Fin de la trayectoria alternativa.

% \paragraph{Trayectoria alternativa A} \label{SUB-M-CU1.1:TA}
% 	El actor no se encuentra usando la aplicación móvil.
% 	\begin{enumerate}[label=A\arabic*.]
% 		\item {[Sistema]} Muestra una notificación al actor como la que se observa en la pantalla \hyperref[fig:sub-m-iu1.1.a]{SUB-M-IU1.1-Confirmar medición (a)}.
% 		\item {[Actor]} Presiona la notificación.
% 		\item {[Sistema]} Continúa en el paso \ref{SUB-M-CU1.1:Pantalla} de la Trayectoria Principal.
% 	\end{enumerate}
% 	Fin de la trayectoria alternativa.
