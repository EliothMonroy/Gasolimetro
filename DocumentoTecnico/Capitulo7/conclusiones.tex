%%%%%%%%%%%%%%%%%%%%%%%%%%%%%%%%%%%%%%%%%%%%%%%%%%%%%%%%%%%%%%%%%%%%%%%%%
%           Capítulo 7: Conclusiones                  %
%%%%%%%%%%%%%%%%%%%%%%%%%%%%%%%%%%%%%%%%%%%%%%%%%%%%%%%%%%%%%%%%%%%%%%%%%
\chapter{Conclusiones}\label{chapter7}
La problemática planteada explicas las fallas existentes en el despacho de gasolina, esta es una problemática real que afecta a una gran cantidad de automovilistas (como lo muestran las encuestas realizadas) los cuales no cuentan con las herramientas necesarias para saber, que gasolineras presentan estás fallas y cuales no. 

El presente trabajo terminal, permite a los automovilistas de la CDMX conocer que gasolineras presentan irregularidades en los litros despachado, con lo cual, dichos automovilistas podrán tomar una decisión informada sobre donde cargar gasolina.

Como se ha presentado en los capítulos anteriores, el trabajo terminal es factible en cada uno de los ámbitos, tanto técnico, como operativo, y económico. Además de que este usa tecnología robusta y de vanguardia para así brindar el mejor servicio a los posibles usuarios. Bajo una arquitectura flexible y robusta como la de microservicios y usando una lenguaje de programación que es estándar en el mercado como Java, el trabajo terminal cuenta con todas las características suficientes para satisfacer los requerimientos funcionales establecidos, y por ende, el objetivo del trabajo terminal.

Del lado del cliente, se realizó una aplicación Android que permite la Geolocalización del usuario, la interacción con su sensor, el manejo de sus datos y la búsqueda de gasolineras a su alrededor. Dicha aplicación se conecta de forma exitosa a nuestro servidor el cual es capaz de almacenar los datos y realizar la clasficiaciones de gasolineras necesarias. De igual forma se desarrollo un prototipo de sensor el cuál permite conocer la cantidad de litros ingresados a un automóvil con un error del 3.9 por ciento, lo cuaĺ esta dentro del rango permitido por la Normas mexicanas para instrumentos de medición.

De esta forma podemos decir que el proyecto terminal cumple con sus objetivos tanto general como específicos brindando a los usuarios finales un "Prototipo de aplicación Móvil para el reporte y medición de la cantidad de combustible que se suministra a un automóvil como el mismo nombre de nuestro TT así lo expresa.


